\documentclass{article}
\usepackage[utf8]{inputenc}
\usepackage[T1]{fontenc}
\usepackage[brazil]{babel}
\usepackage{geometry}
\usepackage{amsmath,amsthm,amsfonts}
\usepackage{graphicx}
%%% BibLaTeX
%\usepackage[style=numeric]{biblatex}
\usepackage[style=ieee]{biblatex}
%\usepackage[style=alphabetic]{biblatex}
%%% \usepackage[style=abnt]{biblatex} % Exige passos adicionais
\addbibresource{aula.bib}
%%% End

\author{Abel Soares Siqueira}
\title{Tutorial de \LaTeX\ para Software Carpentry}
\date{}

\begin{document}
\maketitle

\begin{center}
  \bf
\begin{minipage}{0.7\textwidth}
ATENÇÃO: Este texto foi feito como referência a um tutorial e não é
auto-suficiente.
\end{minipage}
\end{center}

\section{Introdução}

O \LaTeX\ é uma linguagem de marcação feito para textos matemáticos.
Ele funciona a partir de um texto com códigos e comandos, que após
compilação vira um arquivo (PDF, PS, outros) bem formatado.

A extensão tradicional de arquivos de \LaTeX\ é o .tex. Esse .tex é um
arquivo de texto normal, apenas salvo com a extensão .tex. Ele pode ser
aberto em qualquer editor de texto.
Naturalmente, um editor de texto voltado para o \LaTeX\ é melhor.
Alguns já foram sugeridos antes.

As linguagens de marcação, normalmente, são \emph{WYSINWYG - What you
see is not what you get}, isto é, o que você escreve no arquivo .tex e
o que é gerado no pdf são coisas diferentes.
Os processadores de texto não são assim. O Word, Libreoffice e
similares geram arquivos que, quando impressos, devem gerar aquilo que
se está vendo.
Uma vantagem disso, é que você tem mais controle. A desvantagem é que
você tem (muito) mais trabalho.

O \LaTeX\ é a principal linguagem para textos matemáticos. A maior
parte dos períodicos e livros dos últimos 10 anos na área usam apenas o
\LaTeX. Os motivos são simples: O texto fica bonito, e é muito mais
fácil de inserir os objetos importantes do texto: equações.

Essa facilidade tem uma contrapartida: você precisa ``programar'' - na
verdade precisa falar da maneira correta para que o texto possa ser
compilado.

\section{Começando}

Caso não tenha feita, abra o arquivo .tex também.

Antes de começar a escrever o seu texto, você precisa entender alguns
elementos:

\begin{itemize}
  \item Cabeçalho: Você precisa de um cabeçalho indicando
  \begin{itemize}
    \item Que tipo de documento: artigo, livro, etc.;
    \item Que pacotes chamar: idioma, símbolos, gráficos, etc.;
    \item Autor, Título, Data, etc.;
    \item Definições.
  \end{itemize}
  \item Tudo escrito tem que estar entre \verb+\begin{document}+ e
\verb+\end{document}+.
\end{itemize}

O texto básico no \LaTeX\ não tem formatação:

Este
texto
não diferencia
  espaços
     ou
quebra      de     linhas.

Não existe negrito, tamanho de letra, ou a fonte no arquivo .tex.
Tudo isso {\bf pode} ser {\LARGE criado} no PDF através de comandos no
\LaTeX.

Outra coisa que você já deve ter percebido é que o \LaTeX\ que decide o
que vai aonde. Ao escrever o texto não precisamos nos preocupar com
numerações, mudança de página, comprimento da linha (recomendo manter
as linhas no tex curtas, por causa do {\tt git}, no entanto), etc.
Por um lado isso é bom porque você foca apenas no desenvolvimento do
texto. Por outro é ruim porque quando você realmente precisa mudar
algo, dá um certo trabalho. Veremos situações assim em breve.

\section{Ambientes matemáticos}

Fórmulas e símbolos matemáticos exigem alguns pacotes para funcionar.
Algumas coisas específicas requerem outros pacotes, quando existem (a
maior parte existe).
Um exemplo simples é $x^2$. Esse uso é dentro de uma linha.
Você pode quebrar uma linha só
$x^3 - x^2$
que não faz diferença.

Note ainda o espaçamento $x + 2x + x^2$ e
$x+2 x+x ^ 2$.

Quando sozinho numa linha fica assim

$x^2 + x$

Estranho, não?
Para isso existe o próximo comando

$$ x^4 - x^3 $$

ou

\[
  4x(1-x)
\]

Onde colocar esses delimitadores é estilo pessoal.

Com esses dois ambientes já dá pra fazer muita coisa. É preciso
lembrar, ou saber buscar alguns comandos para realmente ter todo o
potencial do \LaTeX, mas quando pegamos o jeito, fica fácil.

\[
  \xi(t) = \sum_{n = 0}^{\infty} \alpha_i
    \frac{\mbox{d}^nf}{\mbox{dx}^n}(a)(t-a)^n
\]

\[
  \mathcal{L}^{-1}\{F(s)G(s)\}(t) =
  f*g = \int_0^t f(\tau) g(t-\tau) \mbox{d}\tau
\]

\[
  [f(g(x))]' = \lim_{h\rightarrow 0} \frac{f(g(x+h))-f(g(x))}{h}
\]

\[
  x = \frac{-b \pm \sqrt{b^2 - 4ac}}{2a}
\]

\[
  z_{k+1} = z_k - [\nabla^2 f(z_k)]^{-1}\nabla f(z_k)
\]

\[
  f(x) = \left\{
  \begin{array}{ll}
    \dfrac{\sin(t)}{t}, & t \neq 0 \\
    1, & t = 0
  \end{array}\right.
\]

\[
  A = \left[\begin{array}{cc}
    a & b \\ c & d
\end{array}
  \right]
  \Rightarrow
  A^{-1} = \frac{1}{ad-bc} \left[\begin{array}{cc}
      d & -b \\ -c & a
  \end{array}
    \right]
\]

\[ \alpha + \beta + \gamma = \zeta + \theta + \delta \]

\[ A + B + \Gamma = Z + \Theta + \Delta \]

\[ \phi = \varphi, \epsilon = \varepsilon \]

Um problema desses ambientes é que eles não tem numeração, o que é
bastante necessário para artigos, teses e afins.

\begin{equation}
  f(x) = x^3
\end{equation}

Para usar, no entanto é preciso dar nomes
\begin{equation}
  \Delta = b^2 - 4ac \label{def.delta}
\end{equation}

A definição de $\Delta$ está na equação \eqref{def.delta}, ou
(\ref{def.delta}).

Outra coisa necessária é quando queremos desenvolver uma conta e
continuar na linha sequinte.
\begin{align}
  \lim_{h\rightarrow 0} \frac{(x+h)^2-x^2}{h}
  & = \lim_{h\rightarrow 0} \frac{2xh + h^2}{h} \\
  & = \lim_{h\rightarrow 0} 2x + h \\
  & = 2x.
\end{align}
Caso queira as quebras de linha, mas não a numeração, faça
\begin{align*}
  ax^2 + bx + c & = a\bigg(x^2 + \frac{b}{a}x\bigg) + c \\
  & = a\bigg(x + \frac{b}{2a}\bigg)^2 -a\frac{b^2}{4a^2} + c \\
  & = a\bigg(x + \frac{b}{2a}\bigg)^2 - \frac{b^2 - 4ac}{4a}
\end{align*}
Se quiser numeração apenas em alguns faça
\begin{align}
  f(x^* + d) & = f(x^*) + \nabla f(x^*)^Td + \frac{1}{2}d^T \nabla^2
f(x^*)d \nonumber \\
& = f(x^*) + \frac{1}{2}d^TAd \nonumber \\
& > f(x^*), \qquad \forall d \neq 0.
\end{align}

Claro, existem outras possibilidades.

\section{Figuras e Tabelas}

Tabela:
\begin{tabular}{r|cl}
  A & tabela & segue \\
  essa & formatação & que é \\ \hline
  bastante & & específica \\ \hline
\end{tabular}

Mas se quisermos mais informações além da tabela, usamos o ambiente
{\tt table}, como mostrado na Tabela \ref{tab:exemplo}.
\begin{table}[ht] % here top
  \centering
\begin{tabular}{|c|c|} \hline
  A & B \\ \hline
  C & D \\ \hline
\end{tabular}
\caption{Exemplo de tabela com table }
\label{tab:exemplo}
\end{table}
Note que a tabela vai pra onde bem quer. {\tt [ht]} é apenas uma
indicação de preferência.

Figura:
\includegraphics[scale=0.3]{ufpr.jpg}

Ambiente da figura mostrado na Figura \ref{fig:exemplo}.
\begin{figure}[!ht] % ! quer dizer por favor.
  \centering
  \includegraphics[width=0.5\textwidth]{ufpr.jpg}
  \caption{Exemplo de figura com figure}
  \label{fig:exemplo}
\end{figure}
As mesmas considerações valem para a figura.

Ambos ambientes são ditos flutuantes, pois a posição deles vai depender
do resto do texto. Em geral, podemos deixar o \LaTeX\ decidir onde é
melhor colocar os flutuantes. Existem pacotes que forçam o
posicionamento, no entanto.
\section{Bibliografia}


A bibliografia no \LaTeX\ é bastante útil. Você precisa definir uma
lista de nomes e códigos, algo do tipo
\begin{verbatim}
\begin{thebibliography}{9}
  \bibitem{bib:exemplo}
    Fulano de Tal,
    \emph{Nome do Livro do Fulano},
    Outras informações, com o formato e ordem que você queira.
\end{thebibliography}
\end{verbatim}
Que vai gerar algo parecido com
\begin{center}
  \fbox{
\begin{minipage}{0.9\textwidth}
Fulano de Tal, \emph{Nome do Livro do Fulano},
Outras informações, com o formato e ordem que você queira.
\end{minipage}
}
\end{center}
Depois basta usar um comando do tipo \verb+\cite{bib:exemplo}+.

No entanto, essa maravilha é trabalhosa. Digamos que você escreve
cinquenta daquelas linhas para seu TCC/Tese seguindo as normas da ABNT.
Mas daí você vai publicar num períodico que segue outras normas.
Então você precisa reescrever esse texto todo.
Para evitar isso, podemos usar o BibTeX.

Veja o arquivo \verb+aula.bib+. \cite{bib:fulano-artigo} é um artigo,
e \cite{bib:fulano-livro}. Note os artigos omitidos e não citados.

\nocite{bib:sem-cit}
%%% Biblatex
\printbibliography
%%% End

\end{document}
